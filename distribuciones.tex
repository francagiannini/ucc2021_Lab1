% Options for packages loaded elsewhere
\PassOptionsToPackage{unicode}{hyperref}
\PassOptionsToPackage{hyphens}{url}
%
\documentclass[
]{article}
\usepackage{amsmath,amssymb}
\usepackage{lmodern}
\usepackage{ifxetex,ifluatex}
\ifnum 0\ifxetex 1\fi\ifluatex 1\fi=0 % if pdftex
  \usepackage[T1]{fontenc}
  \usepackage[utf8]{inputenc}
  \usepackage{textcomp} % provide euro and other symbols
\else % if luatex or xetex
  \usepackage{unicode-math}
  \defaultfontfeatures{Scale=MatchLowercase}
  \defaultfontfeatures[\rmfamily]{Ligatures=TeX,Scale=1}
\fi
% Use upquote if available, for straight quotes in verbatim environments
\IfFileExists{upquote.sty}{\usepackage{upquote}}{}
\IfFileExists{microtype.sty}{% use microtype if available
  \usepackage[]{microtype}
  \UseMicrotypeSet[protrusion]{basicmath} % disable protrusion for tt fonts
}{}
\makeatletter
\@ifundefined{KOMAClassName}{% if non-KOMA class
  \IfFileExists{parskip.sty}{%
    \usepackage{parskip}
  }{% else
    \setlength{\parindent}{0pt}
    \setlength{\parskip}{6pt plus 2pt minus 1pt}}
}{% if KOMA class
  \KOMAoptions{parskip=half}}
\makeatother
\usepackage{xcolor}
\IfFileExists{xurl.sty}{\usepackage{xurl}}{} % add URL line breaks if available
\IfFileExists{bookmark.sty}{\usepackage{bookmark}}{\usepackage{hyperref}}
\hypersetup{
  pdftitle={Distribuciones en R},
  hidelinks,
  pdfcreator={LaTeX via pandoc}}
\urlstyle{same} % disable monospaced font for URLs
\usepackage[margin=1in]{geometry}
\usepackage{color}
\usepackage{fancyvrb}
\newcommand{\VerbBar}{|}
\newcommand{\VERB}{\Verb[commandchars=\\\{\}]}
\DefineVerbatimEnvironment{Highlighting}{Verbatim}{commandchars=\\\{\}}
% Add ',fontsize=\small' for more characters per line
\usepackage{framed}
\definecolor{shadecolor}{RGB}{248,248,248}
\newenvironment{Shaded}{\begin{snugshade}}{\end{snugshade}}
\newcommand{\AlertTok}[1]{\textcolor[rgb]{0.94,0.16,0.16}{#1}}
\newcommand{\AnnotationTok}[1]{\textcolor[rgb]{0.56,0.35,0.01}{\textbf{\textit{#1}}}}
\newcommand{\AttributeTok}[1]{\textcolor[rgb]{0.77,0.63,0.00}{#1}}
\newcommand{\BaseNTok}[1]{\textcolor[rgb]{0.00,0.00,0.81}{#1}}
\newcommand{\BuiltInTok}[1]{#1}
\newcommand{\CharTok}[1]{\textcolor[rgb]{0.31,0.60,0.02}{#1}}
\newcommand{\CommentTok}[1]{\textcolor[rgb]{0.56,0.35,0.01}{\textit{#1}}}
\newcommand{\CommentVarTok}[1]{\textcolor[rgb]{0.56,0.35,0.01}{\textbf{\textit{#1}}}}
\newcommand{\ConstantTok}[1]{\textcolor[rgb]{0.00,0.00,0.00}{#1}}
\newcommand{\ControlFlowTok}[1]{\textcolor[rgb]{0.13,0.29,0.53}{\textbf{#1}}}
\newcommand{\DataTypeTok}[1]{\textcolor[rgb]{0.13,0.29,0.53}{#1}}
\newcommand{\DecValTok}[1]{\textcolor[rgb]{0.00,0.00,0.81}{#1}}
\newcommand{\DocumentationTok}[1]{\textcolor[rgb]{0.56,0.35,0.01}{\textbf{\textit{#1}}}}
\newcommand{\ErrorTok}[1]{\textcolor[rgb]{0.64,0.00,0.00}{\textbf{#1}}}
\newcommand{\ExtensionTok}[1]{#1}
\newcommand{\FloatTok}[1]{\textcolor[rgb]{0.00,0.00,0.81}{#1}}
\newcommand{\FunctionTok}[1]{\textcolor[rgb]{0.00,0.00,0.00}{#1}}
\newcommand{\ImportTok}[1]{#1}
\newcommand{\InformationTok}[1]{\textcolor[rgb]{0.56,0.35,0.01}{\textbf{\textit{#1}}}}
\newcommand{\KeywordTok}[1]{\textcolor[rgb]{0.13,0.29,0.53}{\textbf{#1}}}
\newcommand{\NormalTok}[1]{#1}
\newcommand{\OperatorTok}[1]{\textcolor[rgb]{0.81,0.36,0.00}{\textbf{#1}}}
\newcommand{\OtherTok}[1]{\textcolor[rgb]{0.56,0.35,0.01}{#1}}
\newcommand{\PreprocessorTok}[1]{\textcolor[rgb]{0.56,0.35,0.01}{\textit{#1}}}
\newcommand{\RegionMarkerTok}[1]{#1}
\newcommand{\SpecialCharTok}[1]{\textcolor[rgb]{0.00,0.00,0.00}{#1}}
\newcommand{\SpecialStringTok}[1]{\textcolor[rgb]{0.31,0.60,0.02}{#1}}
\newcommand{\StringTok}[1]{\textcolor[rgb]{0.31,0.60,0.02}{#1}}
\newcommand{\VariableTok}[1]{\textcolor[rgb]{0.00,0.00,0.00}{#1}}
\newcommand{\VerbatimStringTok}[1]{\textcolor[rgb]{0.31,0.60,0.02}{#1}}
\newcommand{\WarningTok}[1]{\textcolor[rgb]{0.56,0.35,0.01}{\textbf{\textit{#1}}}}
\usepackage{graphicx}
\makeatletter
\def\maxwidth{\ifdim\Gin@nat@width>\linewidth\linewidth\else\Gin@nat@width\fi}
\def\maxheight{\ifdim\Gin@nat@height>\textheight\textheight\else\Gin@nat@height\fi}
\makeatother
% Scale images if necessary, so that they will not overflow the page
% margins by default, and it is still possible to overwrite the defaults
% using explicit options in \includegraphics[width, height, ...]{}
\setkeys{Gin}{width=\maxwidth,height=\maxheight,keepaspectratio}
% Set default figure placement to htbp
\makeatletter
\def\fps@figure{htbp}
\makeatother
\setlength{\emergencystretch}{3em} % prevent overfull lines
\providecommand{\tightlist}{%
  \setlength{\itemsep}{0pt}\setlength{\parskip}{0pt}}
\setcounter{secnumdepth}{-\maxdimen} % remove section numbering
\ifluatex
  \usepackage{selnolig}  % disable illegal ligatures
\fi

\title{Distribuciones en R}
\author{}
\date{\vspace{-2.5em}}

\begin{document}
\maketitle

\hypertarget{computo-de-probabilidades}{%
\section{Computo de probabilidades}\label{computo-de-probabilidades}}

Para todas las variables aleatorias que hemos mencionado hasta ahora (¡y
muchas más!), R ha incorporado capacidades de cálculo de probabilidades.
La sintaxis se divide en dos partes: la raíz y el prefijo. La raíz
determina de qué variable aleatoria estamos hablando, y aquí están los
nombres de las que hemos cubierto hasta ahora:

\begin{itemize}
\tightlist
\item
  \texttt{binom} binomial
\item
  \texttt{geom} geometrica
\item
  \texttt{pois} Poisson
\item
  \texttt{unif} uniforme
\item
  \texttt{exp} nexponencial
\item
  \texttt{norm} normal
\end{itemize}

Y los prefijos disponibles son los siguientes:

\begin{itemize}
\tightlist
\item
  \texttt{p} computa la \textbf{distribución acumulada}
\item
  \texttt{d} computa la \textbf{distribución de probabilidad} o
  \textbf{distribución de densidad}\\
\item
  \texttt{r} muestrea
\item
  \texttt{q} es la distribución cuantil
\end{itemize}

Por el momento nos vamos a enfocar en \texttt{p} y \texttt{d}

Por ejemplo si \(X\) es una variable aleatoria que se distribuye
binomial con \(n=10\) y \(p=0.3\) y queremos computar \(P(X=5)\)
entonces:

\begin{Shaded}
\begin{Highlighting}[]
\FunctionTok{dbinom}\NormalTok{(}\DecValTok{5}\NormalTok{, }\DecValTok{10}\NormalTok{, }\FloatTok{0.3}\NormalTok{)}
\end{Highlighting}
\end{Shaded}

\begin{verbatim}
## [1] 0.1029193
\end{verbatim}

Recordemos que siempre podemos pedir la ayuda de la función para ver
cuales son sus parámetros

Si queremos ahora calcular \(P(1\leqslant X \leqslant 5)\) podemos
hacerlo de dos maneras:

\emph{Opción 1}

\[P(X = 1) + P(X = 2) + P(X = 3) + P(X = 4) + P(X = 5)\]

\begin{Shaded}
\begin{Highlighting}[]
\FunctionTok{dbinom}\NormalTok{(}\DecValTok{5}\NormalTok{, }\DecValTok{10}\NormalTok{, }\FloatTok{0.3}\NormalTok{) }\SpecialCharTok{+}
  \FunctionTok{dbinom}\NormalTok{(}\DecValTok{4}\NormalTok{, }\DecValTok{10}\NormalTok{, }\FloatTok{0.3}\NormalTok{) }\SpecialCharTok{+}
  \FunctionTok{dbinom}\NormalTok{(}\DecValTok{3}\NormalTok{, }\DecValTok{10}\NormalTok{, }\FloatTok{0.3}\NormalTok{) }\SpecialCharTok{+}
  \FunctionTok{dbinom}\NormalTok{(}\DecValTok{2}\NormalTok{, }\DecValTok{10}\NormalTok{, }\FloatTok{0.3}\NormalTok{) }\SpecialCharTok{+}
  \FunctionTok{dbinom}\NormalTok{(}\DecValTok{1}\NormalTok{, }\DecValTok{10}\NormalTok{, }\FloatTok{0.3}\NormalTok{)}
\end{Highlighting}
\end{Shaded}

\begin{verbatim}
## [1] 0.9244035
\end{verbatim}

Para no volvernos viejos tan rápido podemos usar la opción de
\texttt{dbinom} también permite vectores como argumentos, por lo que
podemos calcular todas las probabilidades necesarias de esta manera:

\emph{Opción 2}

\begin{Shaded}
\begin{Highlighting}[]
\FunctionTok{dbinom}\NormalTok{(}\DecValTok{1}\SpecialCharTok{:}\DecValTok{5}\NormalTok{, }\DecValTok{10}\NormalTok{, }\FloatTok{0.3}\NormalTok{)}
\end{Highlighting}
\end{Shaded}

\begin{verbatim}
## [1] 0.1210608 0.2334744 0.2668279 0.2001209 0.1029193
\end{verbatim}

\begin{Shaded}
\begin{Highlighting}[]
\FunctionTok{sum}\NormalTok{(}\FunctionTok{dbinom}\NormalTok{(}\DecValTok{1}\SpecialCharTok{:}\DecValTok{5}\NormalTok{, }\DecValTok{10}\NormalTok{ , }\FloatTok{0.3}\NormalTok{))}
\end{Highlighting}
\end{Shaded}

\begin{verbatim}
## [1] 0.9244035
\end{verbatim}

\hypertarget{algunos-ejemplos-con-covid-19}{%
\subsection{Algunos ejemplos con COVID
19}\label{algunos-ejemplos-con-covid-19}}

Para comenzar vamos a obtener una serie de datos de la página oficial de
la OMS o \href{https://covid19.who.int/info}{WHO}

Vamos a tratar de responder algunos interrogantes respecto a la
incidencia de la enfermedad

Trabajaremos con los datos totales y de los últimos 7 días, debemos
elegir Argentina y un país de cada region

Definimos Prevalencia como la proporción de individuos de un grupo o una
población (en medicina, persona), que presentan una característica o
evento determinado (en medicina, enfermedades). Por lo general, se
expresa como una fracción, un porcentaje o un número de casos por cada
10 000 o 100.000 personas.

Mientras que la incidencia es el número de casos nuevos de una
enfermedad en una población determinada y en un periodo determinado.

Definimos la Incidencia para un periodo de tiempo dado como:

\[Incidencia= \frac{numero.de.personas.enfermas}{numero.de.habitantes}\]

La prevalencia es un término que significa estar extendido y es distinto
de la incidencia .

La prevalencia es una medida de todos los individuos (casos) afectados
por la enfermedad en un momento determinado, mientras que la incidencia
es una medida del número de nuevos individuos (casos) que contraen una
enfermedad durante un período de tiempo particular.

La prevalencia responde a ``¿Cuántas personas tienen esta enfermedad en
este momento?'' o ``¿Cuántas personas han tenido esta enfermedad durante
este período de tiempo?''. La incidencia responde a ``¿Cuántas personas
adquirieron la enfermedad durante un período de tiempo específico?''.
Sin embargo, matemáticamente, la prevalencia es proporcional al producto
de la incidencia y la duración promedio de la enfermedad.

\[ Prevalencia  = Incidencia* duración\]

¿Cuál es la probabilidad de que una persona contraiga la enfermdad en
los últimos 7 días?

Asumimos una distribución Bernoulli o binomial con n=1

\[P(X=x)=p^x (1-p)^{(n-x)}\] con

\[𝑥={0,1}\]

Aqui \(p\) tomará el valor de la incidencia media y \(x\) sera 1

Entonces por ejemplo para el mundo con una incidencia de
\(\frac{58.848}{100000}=0.00058848\) :

\begin{Shaded}
\begin{Highlighting}[]
\FunctionTok{dbinom}\NormalTok{(}\AttributeTok{x=}\DecValTok{1}\NormalTok{, }\AttributeTok{size=}\DecValTok{1}\NormalTok{, }\AttributeTok{prob=}\FloatTok{58.848}\SpecialCharTok{/}\DecValTok{100000}\NormalTok{)}
\end{Highlighting}
\end{Shaded}

\begin{verbatim}
## [1] 0.00058848
\end{verbatim}

Obviamente observando la formula podiamos darnos cuenta que ese valor
iba a ser igual a \(p\)

¿Ahora, cuál es la probabilidad de que un integrante de una burbuja de
15 personas contraiga la enfermdad en los últimos 7 días?

\begin{Shaded}
\begin{Highlighting}[]
\FunctionTok{dbinom}\NormalTok{(}\AttributeTok{x=}\DecValTok{1}\NormalTok{, }\AttributeTok{size=}\DecValTok{15}\NormalTok{, }\AttributeTok{prob=}\NormalTok{(}\FloatTok{58.848}\SpecialCharTok{/}\DecValTok{100000}\NormalTok{))}
\end{Highlighting}
\end{Shaded}

\begin{verbatim}
## [1] 0.008754753
\end{verbatim}

y dos?

\begin{Shaded}
\begin{Highlighting}[]
\FunctionTok{dbinom}\NormalTok{(}\AttributeTok{x=}\DecValTok{2}\NormalTok{, }\AttributeTok{size=}\DecValTok{15}\NormalTok{, }\AttributeTok{prob=}\FloatTok{58.848}\SpecialCharTok{/}\DecValTok{100000}\NormalTok{)}
\end{Highlighting}
\end{Shaded}

\begin{verbatim}
## [1] 3.608521e-05
\end{verbatim}

¿Qué pasa en Argentina y en los paises seleccionados de cada región?

Comparar en Argentina \(p\) total, de los ultimos 7 días y de las
últimas 24 hs.

\hypertarget{variables-continuas}{%
\subsection{Variables continuas}\label{variables-continuas}}

Ahora, siendo \(X\) una v.a. de media 1 (\(\mu=1\)) y desvío estandar 5
(\(\sigma = 5\)) y queremos calcular \(P(X\leqslant 1)\)

Como se trata de una distribución de variables continuas es complejo
calcular probabilidades como el ejemplo anterior, aquí tenemos que
integrar. Entoncess:

\[\int_{-\infty}^1 {\text { N(x, 1, 5)}} \, dx\] Pero, la probabilidad
que estamos tratando de calcular podemos computarla a traves de \(F(x)\)
usando \texttt{pnorm}.

\begin{Shaded}
\begin{Highlighting}[]
\FunctionTok{pnorm}\NormalTok{(}\DecValTok{1}\NormalTok{,}\DecValTok{1}\NormalTok{,}\DecValTok{5}\NormalTok{)}
\end{Highlighting}
\end{Shaded}

\begin{verbatim}
## [1] 0.5
\end{verbatim}

Si queremos calcular \(P(1 \le X \le 5)\) Aquí, calculamos la
probabilidad de que \(X\) sea menor o igual a cinco, y restamos la
probabilidad de que \(X\) sea menor que 1. Que es lo mismo que:

\[P(X \le 5) - P(X \le 1) \]

\begin{Shaded}
\begin{Highlighting}[]
\FunctionTok{pnorm}\NormalTok{(}\DecValTok{5}\NormalTok{,}\DecValTok{1}\NormalTok{,}\DecValTok{5}\NormalTok{) }\SpecialCharTok{{-}} \FunctionTok{pnorm}\NormalTok{(}\DecValTok{1}\NormalTok{,}\DecValTok{1}\NormalTok{,}\DecValTok{5}\NormalTok{)}
\end{Highlighting}
\end{Shaded}

\begin{verbatim}
## [1] 0.2881446
\end{verbatim}

\emph{Ejemplo}

Debemos elegir siete mujeres al azar de una universidad para formar una
versión inicial de un video juego de basquet femenino. Las alturas de
las mujeres en se distribuyen normalmente con una media de 163.83 cm y
una desviación estándar de 5.715 cm. ¿Cuál es la probabilidad de que 3 o
más de las mujeres midan 172.72 cm o más?

Para computar esta probabilidad, primero determinamos la probabilidad de
que una sola mujer seleccionada al azar mida 172.72 cm o más alta. Sea
\(X\) una variable aleatoria normal con media 162.56 y desviación
estándar 5.715. Calculamos \(P(X \ge 172.72)\) usando \texttt{pnorm}:

\begin{Shaded}
\begin{Highlighting}[]
\FunctionTok{pnorm}\NormalTok{(}\FloatTok{172.72}\NormalTok{, }\FloatTok{162.56}\NormalTok{, }\FloatTok{5.715}\NormalTok{, }\AttributeTok{lower.tail =} \ConstantTok{FALSE}\NormalTok{)}
\end{Highlighting}
\end{Shaded}

\begin{verbatim}
## [1] 0.03772018
\end{verbatim}

Ahora, tenemos que calcular la probabilidad de que 3 o más de las 7
mujeres midan 172.72 cm o más. Dado que la población de todas las
mujeres en una universidad es mucho mayor que 7, el número de mujeres en
la configuracón inicial que tienen 172.72 cm o más es binomial con
\(n = 7\) y \(0.03772018\), que calculamos en el paso anterior.
Entonces, calculamos la probabilidad de que al menos 3 mujeres midan
172.72 cm como:

\begin{Shaded}
\begin{Highlighting}[]
\FunctionTok{sum}\NormalTok{(}\FunctionTok{dbinom}\NormalTok{(}\DecValTok{3}\SpecialCharTok{:}\DecValTok{7}\NormalTok{, }\DecValTok{7}\NormalTok{, }\FloatTok{0.03772018}\NormalTok{))}
\end{Highlighting}
\end{Shaded}

\begin{verbatim}
## [1] 0.001675265
\end{verbatim}

Por lo tanto, hay aproximadamente un \(0.17\) por ciento de
posibilidades de que al menos tres mujeres midan 172.72 cm o más.

\pagebreak

\emph{Graficar funciones con ggplot2}

Para graficar la función de densidad del ejemplo anterior, vamos a
generar una serie de datos (vector) de datos con los valores de \(x\) y
los valores de \(y\) que queremos representar (usando la función
\texttt{dnorm()}) y a partir de ese vector vamos a graficar usando
\texttt{geom\_line()}, de la siguiente manera.

\begin{Shaded}
\begin{Highlighting}[]
\FunctionTok{library}\NormalTok{(ggplot2)}

\NormalTok{xvals }\OtherTok{\textless{}{-}} \FunctionTok{seq}\NormalTok{(}\DecValTok{120}\NormalTok{,}\DecValTok{200}\NormalTok{,}\DecValTok{1}\NormalTok{)}
\NormalTok{plotdata }\OtherTok{\textless{}{-}} \FunctionTok{data.frame}\NormalTok{(}\AttributeTok{x =}\NormalTok{ xvals, }\AttributeTok{y =} \FunctionTok{dnorm}\NormalTok{(xvals, }\FloatTok{162.56}\NormalTok{, }\FloatTok{5.715}\NormalTok{))}
\FunctionTok{ggplot}\NormalTok{(plotdata, }\FunctionTok{aes}\NormalTok{(}\AttributeTok{x =}\NormalTok{ x, }\AttributeTok{y =}\NormalTok{ y)) }\SpecialCharTok{+}
  \FunctionTok{geom\_line}\NormalTok{()}\SpecialCharTok{+}
  \FunctionTok{labs}\NormalTok{(}\AttributeTok{x=}\StringTok{"altura (cm)"}\NormalTok{, }\AttributeTok{y=} \StringTok{"f(x)"}\NormalTok{, }\AttributeTok{title=}\StringTok{"Distribución Normal(162.56, 5.715)"}\NormalTok{)}
\end{Highlighting}
\end{Shaded}

\includegraphics{distribuciones_files/figure-latex/unnamed-chunk-11-1.pdf}

\pagebreak

De la misma manera si se trata de una distribución binomial

\begin{Shaded}
\begin{Highlighting}[]
\NormalTok{xvals }\OtherTok{\textless{}{-}} \DecValTok{0}\SpecialCharTok{:}\DecValTok{7}
\NormalTok{plotdata }\OtherTok{\textless{}{-}} \FunctionTok{data.frame}\NormalTok{(}\AttributeTok{x =}\NormalTok{ xvals, }\AttributeTok{y =} \FunctionTok{dbinom}\NormalTok{(xvals, }\DecValTok{7}\NormalTok{, }\FloatTok{0.03772018}\NormalTok{))}
\FunctionTok{ggplot}\NormalTok{(plotdata, }\FunctionTok{aes}\NormalTok{(x, y)) }\SpecialCharTok{+} \FunctionTok{geom\_bar}\NormalTok{(}\AttributeTok{stat =} \StringTok{"identity"}\NormalTok{)}\SpecialCharTok{+}
  \FunctionTok{labs}\NormalTok{(}\AttributeTok{x=}\StringTok{"altura (cm)"}\NormalTok{, }\AttributeTok{y=} \StringTok{"f(X)"}\NormalTok{, }\AttributeTok{title=}\StringTok{"Distribución Binomial(7; 0,037)"}\NormalTok{)}
\end{Highlighting}
\end{Shaded}

\includegraphics{distribuciones_files/figure-latex/unnamed-chunk-12-1.pdf}

\emph{Otras opciones para variables continuas }

\begin{Shaded}
\begin{Highlighting}[]
\FunctionTok{curve}\NormalTok{(}\FunctionTok{dgamma}\NormalTok{(x, }\AttributeTok{scale=}\FloatTok{1.5}\NormalTok{, }\AttributeTok{shape=}\DecValTok{2}\NormalTok{),}\AttributeTok{from=}\DecValTok{0}\NormalTok{, }\AttributeTok{to=}\DecValTok{15}\NormalTok{, }\AttributeTok{main=}\StringTok{"distribución Gamma"}\NormalTok{)}
\end{Highlighting}
\end{Shaded}

\includegraphics{distribuciones_files/figure-latex/unnamed-chunk-13-1.pdf}

\begin{Shaded}
\begin{Highlighting}[]
\FunctionTok{curve}\NormalTok{(}\FunctionTok{dunif}\NormalTok{(x,}\AttributeTok{min=}\DecValTok{0}\NormalTok{,}\AttributeTok{max=}\DecValTok{10}\NormalTok{),}\AttributeTok{from=}\DecValTok{0}\NormalTok{,}\AttributeTok{to=}\DecValTok{10}\NormalTok{, }\AttributeTok{main=}\StringTok{"distribución uniforme"}\NormalTok{)}
\end{Highlighting}
\end{Shaded}

\includegraphics{distribuciones_files/figure-latex/unnamed-chunk-13-2.pdf}

\begin{Shaded}
\begin{Highlighting}[]
\FunctionTok{curve}\NormalTok{(}\FunctionTok{dnorm}\NormalTok{(x,}\AttributeTok{m=}\DecValTok{10}\NormalTok{,}\AttributeTok{sd=}\DecValTok{2}\NormalTok{),}\AttributeTok{from=}\DecValTok{0}\NormalTok{,}\AttributeTok{to=}\DecValTok{20}\NormalTok{,}\AttributeTok{main=}\StringTok{"distribución Normal"}\NormalTok{)}
\end{Highlighting}
\end{Shaded}

\includegraphics{distribuciones_files/figure-latex/unnamed-chunk-13-3.pdf}

\emph{Ejemplo Salario neto mensual}

Veamos el comportamiento de la variable salario neto

\begin{Shaded}
\begin{Highlighting}[]
\FunctionTok{head}\NormalTok{(salario\_neto\_gen)}
\end{Highlighting}
\end{Shaded}

\begin{verbatim}
## # A tibble: 6 x 3
##     neto `Me identifico` genero   
##    <dbl> <chr>           <fct>    
## 1  90000 Varón Cis       Varón Cis
## 2 109000 Varón Cis       Varón Cis
## 3  39259 Varón Cis       Varón Cis
## 4  91713 Varón Cis       Varón Cis
## 5 137700 Varón Cis       Varón Cis
## 6  38500 Varón Cis       Varón Cis
\end{verbatim}

Función de densidad

\begin{Shaded}
\begin{Highlighting}[]
\FunctionTok{plot}\NormalTok{(}\FunctionTok{density}\NormalTok{(salario\_neto\_gen}\SpecialCharTok{$}\NormalTok{neto), }\AttributeTok{main=}\StringTok{\textquotesingle{}\textquotesingle{}}\NormalTok{, }\AttributeTok{lwd=}\DecValTok{5}\NormalTok{, }\AttributeTok{las=}\DecValTok{1}\NormalTok{,}
     \AttributeTok{xlab=}\StringTok{\textquotesingle{}Salario\textquotesingle{}}\NormalTok{, }\AttributeTok{ylab=}\StringTok{\textquotesingle{}Densidad\textquotesingle{}}\NormalTok{)}
\end{Highlighting}
\end{Shaded}

\includegraphics{distribuciones_files/figure-latex/unnamed-chunk-16-1.pdf}

Con ggplot2 por genero

\begin{Shaded}
\begin{Highlighting}[]
\FunctionTok{ggplot}\NormalTok{(salario\_neto\_gen, }\FunctionTok{aes}\NormalTok{(}\AttributeTok{x=}\NormalTok{neto)) }\SpecialCharTok{+} 
  \FunctionTok{geom\_density}\NormalTok{(}\FunctionTok{aes}\NormalTok{(}\AttributeTok{group=}\NormalTok{genero, }\AttributeTok{fill=}\NormalTok{genero), }\AttributeTok{alpha=}\FloatTok{0.5}\NormalTok{)}
\end{Highlighting}
\end{Shaded}

\includegraphics{distribuciones_files/figure-latex/unnamed-chunk-17-1.pdf}

Función de distribución acumulada

\begin{Shaded}
\begin{Highlighting}[]
\NormalTok{F }\OtherTok{\textless{}{-}} \FunctionTok{ecdf}\NormalTok{(salario\_neto\_gen}\SpecialCharTok{$}\NormalTok{neto)}
\FunctionTok{plot}\NormalTok{(F, }\AttributeTok{main=}\StringTok{\textquotesingle{}\textquotesingle{}}\NormalTok{, }\AttributeTok{xlab=}\StringTok{\textquotesingle{}Peso (Kg)\textquotesingle{}}\NormalTok{, }\AttributeTok{ylab=}\StringTok{\textquotesingle{}F(x)\textquotesingle{}}\NormalTok{, }\AttributeTok{cex=}\FloatTok{0.5}\NormalTok{, }\AttributeTok{las=}\DecValTok{1}\NormalTok{)}
\end{Highlighting}
\end{Shaded}

\includegraphics{distribuciones_files/figure-latex/unnamed-chunk-18-1.pdf}

Con ggplot2 por genero

\begin{Shaded}
\begin{Highlighting}[]
\FunctionTok{ggplot}\NormalTok{(salario\_neto\_gen, }\FunctionTok{aes}\NormalTok{(neto, }\AttributeTok{fill=}\NormalTok{genero, }\AttributeTok{col=}\NormalTok{genero))}\SpecialCharTok{+}
  \FunctionTok{geom\_step}\NormalTok{(}\AttributeTok{stat=}\StringTok{"ecdf"}\NormalTok{) }\SpecialCharTok{+}
  \FunctionTok{ylab}\NormalTok{(}\StringTok{"Frecuencia acumulada"}\NormalTok{)}\SpecialCharTok{+}
  \FunctionTok{xlab}\NormalTok{(}\StringTok{"Salario neto mensual"}\NormalTok{)}
\end{Highlighting}
\end{Shaded}

\includegraphics{distribuciones_files/figure-latex/unnamed-chunk-19-1.pdf}

Calcular la probabilidad de que una persona de la industria del software
cobre un sueldo inferior al salario minimo vital y movil.

SMVM = 21600

\begin{Shaded}
\begin{Highlighting}[]
\FunctionTok{F}\NormalTok{(}\DecValTok{21600}\NormalTok{)}
\end{Highlighting}
\end{Shaded}

\begin{verbatim}
## [1] 0.01651572
\end{verbatim}

\hypertarget{tcl}{%
\section{TCL}\label{tcl}}

\hypertarget{media-muestral-del-salario}{%
\subsection{Media muestral del
salario}\label{media-muestral-del-salario}}

Ahora muestreamos 10 individuos y calculamos la media del salario sobre
esos 10 individuos

Muestreo una vez una muestra de tamaño 10

\begin{Shaded}
\begin{Highlighting}[]
\NormalTok{muestra\_size10\_1 }\OtherTok{\textless{}{-}} \FunctionTok{sample}\NormalTok{(salario\_neto\_gen}\SpecialCharTok{$}\NormalTok{neto, }\AttributeTok{size=}\DecValTok{10}\NormalTok{)}
\end{Highlighting}
\end{Shaded}

Calculo la media

\begin{Shaded}
\begin{Highlighting}[]
\FunctionTok{mean}\NormalTok{(muestra\_size10\_1)}
\end{Highlighting}
\end{Shaded}

\begin{verbatim}
## [1] 81075.9
\end{verbatim}

Repito el proceso 100 veces

\begin{Shaded}
\begin{Highlighting}[]
\NormalTok{medias\_muestrales }\OtherTok{\textless{}{-}} \FunctionTok{replicate}\NormalTok{(}\DecValTok{100}\NormalTok{, }\FunctionTok{mean}\NormalTok{(}\FunctionTok{sample}\NormalTok{(salario\_neto\_gen}\SpecialCharTok{$}\NormalTok{neto, }\AttributeTok{size=}\DecValTok{10}\NormalTok{)))}

\FunctionTok{plot}\NormalTok{(}\FunctionTok{density}\NormalTok{(medias\_muestrales), }\AttributeTok{main=}\StringTok{\textquotesingle{}\textquotesingle{}}\NormalTok{, }\AttributeTok{lwd=}\DecValTok{5}\NormalTok{, }\AttributeTok{las=}\DecValTok{1}\NormalTok{,}
     \AttributeTok{xlab=}\StringTok{\textquotesingle{}Media muestral Salario n=10 replicas=100\textquotesingle{}}\NormalTok{, }\AttributeTok{ylab=}\StringTok{\textquotesingle{}Densidad\textquotesingle{}}\NormalTok{)}
\end{Highlighting}
\end{Shaded}

\includegraphics{distribuciones_files/figure-latex/unnamed-chunk-23-1.pdf}

\begin{Shaded}
\begin{Highlighting}[]
\FunctionTok{mean}\NormalTok{(medias\_muestrales)}
\end{Highlighting}
\end{Shaded}

\begin{verbatim}
## [1] 100781.6
\end{verbatim}

\begin{Shaded}
\begin{Highlighting}[]
\FunctionTok{sd}\NormalTok{(medias\_muestrales)}
\end{Highlighting}
\end{Shaded}

\begin{verbatim}
## [1] 19644.05
\end{verbatim}

Ahora lo repito 1000 veces y 100000 veces

\begin{Shaded}
\begin{Highlighting}[]
\NormalTok{medias\_muestrales }\OtherTok{\textless{}{-}} \FunctionTok{replicate}\NormalTok{(}\DecValTok{1000}\NormalTok{, }\FunctionTok{mean}\NormalTok{(}\FunctionTok{sample}\NormalTok{(salario\_neto\_gen}\SpecialCharTok{$}\NormalTok{neto, }\AttributeTok{size=}\DecValTok{10}\NormalTok{)))}

\FunctionTok{plot}\NormalTok{(}\FunctionTok{density}\NormalTok{(medias\_muestrales), }\AttributeTok{main=}\StringTok{\textquotesingle{}\textquotesingle{}}\NormalTok{, }\AttributeTok{lwd=}\DecValTok{5}\NormalTok{, }\AttributeTok{las=}\DecValTok{1}\NormalTok{,}
     \AttributeTok{xlab=}\StringTok{\textquotesingle{}Media muestral Salario n=10 replicas=1000\textquotesingle{}}\NormalTok{, }\AttributeTok{ylab=}\StringTok{\textquotesingle{}Densidad\textquotesingle{}}\NormalTok{)}
\end{Highlighting}
\end{Shaded}

\includegraphics{distribuciones_files/figure-latex/unnamed-chunk-24-1.pdf}

\begin{Shaded}
\begin{Highlighting}[]
\FunctionTok{mean}\NormalTok{(medias\_muestrales)}
\end{Highlighting}
\end{Shaded}

\begin{verbatim}
## [1] 101533.3
\end{verbatim}

\begin{Shaded}
\begin{Highlighting}[]
\FunctionTok{sd}\NormalTok{(medias\_muestrales)}
\end{Highlighting}
\end{Shaded}

\begin{verbatim}
## [1] 19781.88
\end{verbatim}

\begin{Shaded}
\begin{Highlighting}[]
\NormalTok{medias\_muestrales }\OtherTok{\textless{}{-}} \FunctionTok{replicate}\NormalTok{(}\DecValTok{1000}\NormalTok{, }\FunctionTok{mean}\NormalTok{(}\FunctionTok{sample}\NormalTok{(salario\_neto\_gen}\SpecialCharTok{$}\NormalTok{neto, }\AttributeTok{size=}\DecValTok{10}\NormalTok{)))}

\FunctionTok{plot}\NormalTok{(}\FunctionTok{density}\NormalTok{(medias\_muestrales), }\AttributeTok{main=}\StringTok{\textquotesingle{}\textquotesingle{}}\NormalTok{, }\AttributeTok{lwd=}\DecValTok{5}\NormalTok{, }\AttributeTok{las=}\DecValTok{1}\NormalTok{,}
     \AttributeTok{xlab=}\StringTok{\textquotesingle{}Media muestral Salario n=10 replicas=100000\textquotesingle{}}\NormalTok{, }\AttributeTok{ylab=}\StringTok{\textquotesingle{}Densidad\textquotesingle{}}\NormalTok{)}
\end{Highlighting}
\end{Shaded}

\includegraphics{distribuciones_files/figure-latex/unnamed-chunk-24-2.pdf}

\begin{Shaded}
\begin{Highlighting}[]
\FunctionTok{mean}\NormalTok{(medias\_muestrales)}
\end{Highlighting}
\end{Shaded}

\begin{verbatim}
## [1] 102212.5
\end{verbatim}

\begin{Shaded}
\begin{Highlighting}[]
\FunctionTok{sd}\NormalTok{(medias\_muestrales)}
\end{Highlighting}
\end{Shaded}

\begin{verbatim}
## [1] 19960.34
\end{verbatim}

Repetir el proceso para un tamaño mestral de 20 y un tamaño mestral de
100, calculando media y desvio estandar para cada caso. Construir una
tabla y concluir.

\end{document}
